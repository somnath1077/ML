\documentclass[xcolor=table]{beamer}
\usepackage[utf8]{inputenc}
\usepackage{default}
\usepackage{xspace,setspace}
\usepackage{amsmath,amsthm,amssymb}
\usepackage{ellipsis}
\usepackage[pdftex]{epsfig}
\usepackage{listliketab}
\usepackage[table]{xcolor}
\usepackage{booktabs}
\usepackage{amsmath}

\usepackage{pgf}
\usepackage{tikz}
\usetikzlibrary{arrows,automata}
\usetikzlibrary{graphs}
\tikzset{
     mainNode/.style =
        { circle
        , draw
        %, fill=blue!20,
        %, font=\sffamily\Large\bfseries
        }
}

\usetheme{AnnArbor}
%\usetheme{Berlin}
%\usetheme{Bergen}
%\usetheme{Antibes}
%\usetheme{Goettingen}
%\usetheme{Warsaw}
%\usetheme{Darmstadt}
%\usetheme{JuanLesPins}
%\setbeamertemplate{navigation symbols}{}
%\usecolortheme{beaver}
%\usecolortheme{rose}
%\usecolortheme{seagull}
%\usecolortheme{dove}
%\usecolortheme{seahorse}
%\usecolortheme{crane}
%\usepackage{color,colortbl}
%\usepackage{texnansi}
%\usepackage{marvosym}
%\usepackage{comment}


%\setbeamertemplate{frametitle}
%{\begin{centering}\smallskip
%   \insertframetitle\par
%   \smallskip\end{centering}}
%\setbeamertemplate{itemize item}{$\bullet$}
%\setbeamertemplate{navigation symbols}{}
%\setbeamertemplate{footline}[text line]{%
%    \hfill\strut{%
%        \scriptsize\sf\color{black!60}%
%        \quad\insertframenumber
%    }%
%    \hfill
%}

% Define some colors:

\definecolor{DarkFern}{HTML}{407428}
\definecolor{DarkCharcoal}{HTML}{4D4944}
\colorlet{Fern}{DarkFern!85!white}
\colorlet{Charcoal}{DarkCharcoal!85!white}
\colorlet{LightCharcoal}{Charcoal!50!white}
\colorlet{AlertColor}{orange!80!black}
\colorlet{DarkRed}{red!70!black}
\colorlet{DarkBlue}{olive!70!black}
\colorlet{DarkGreen}{green!70!black}
\definecolor{brickred}{RGB}{132,31,39}

% Use the colors:

%\setbeamercolor{title}{fg=Fern}
%\setbeamercolor{frametitle}{fg=Fern}
%\setbeamercolor{section title}{fg=Fern}
%\setbeamercolor{section in toc}{fg=Fern}
%\setbeamercolor{section name}{fg=Fern}
%\setbeamercolor{section in head/foot}{fg=Fern}
%\setbeamercolor{subsection title}{fg=Fern}
%\setbeamercolor{author}{fg=Fern}
%\setbeamercolor{normal text}{fg=Charcoal}
%\setbeamercolor{block title}{fg=black,bg=Fern!25!white}
%\setbeamercolor{block body}{fg=black,bg=Fern!25!white}
%\setbeamercolor{alerted text}{fg=AlertColor}
%\setbeamercolor{itemize item}{fg=Charcoal}

%\definecolor{bottomcolour}{rgb}{0.32,0.3,0.38}
%\definecolor{middlecolour}{rgb}{0.08,0.08,0.16}
\definecolor{tcsyellow}{RGB}{253,255,102}
\definecolor{tcsolive}{RGB}{77,147,191}
\definecolor{tcsolivemedium}{RGB}{147,177,210}
\definecolor{tcsolivelight}{RGB}{235,239,252}

\definecolor{skyolive}{rgb}{0.2,0.6,1}
\definecolor{darkolive}{rgb}{0.1,0.1,0.6}
\definecolor{darkred}{rgb}{1,0.2,0.1}
\definecolor{darkgreen}{rgb}{0.5,0.8,0.4}
\definecolor{Olive}{rgb}{0,0.3,0}
\definecolor{seagreen}{rgb}{0.3,0.9,0.6}
\definecolor{olive}{cmyk}{0.8,0.1,0.95,0.40}
\definecolor{golden}{cmyk}{0.0,0.25,0.85,0.15}
\definecolor{darkgolden}{cmyk}{0.0,0.27,0.94,0.07}
\definecolor{orange}{cmyk}{0.0,0.35,1.0,0.07}
\definecolor{orange2}{cmyk}{0.0,0.6,1.0, 0.0}
\definecolor{pecan}{cmyk}{0.0,0.37,0.80,0.12}
\definecolor{cadmium}{cmyk}{0.0,0.40,0.93,0.00}
\definecolor{snake}{cmyk}{0.8,0.1,0.95,0.60}
\definecolor{tcsyellow}{RGB}{253,255,102}
\definecolor{tcsolive}{RGB}{77,147,191}
\definecolor{tcsolivemedium}{RGB}{147,177,210}
\definecolor{tcsolivelight}{RGB}{235,239,252}

\definecolor{LRed}{rgb}{1,.8,.8}
\definecolor{MRed}{rgb}{1,.6,.6}
\definecolor{HRed}{rgb}{1,.2,.2}

\newtheorem{defn}{Definition}
\newtheorem{asf}{ASF Inputs}
\newtheorem{claim}{Claim}

\setbeamercolor{item projected}{bg=darkred}
\setbeamertemplate{enumerate items}[orange2]
\setbeamercolor{frametitle}{fg=white,bg=orange2}
\setbeamercolor{title}{fg=white,bg=orange2}
 
\usetheme{boxes}
\setbeamertemplate{blocks}[rounded][shadow=false]
\setbeamertemplate{itemize item}{\color{orange2}$\blacktriangleright$}
\setbeamertemplate{itemize subitem}{\color{orange2}$\blacktriangleright$}
 
\setbeamercolor{block title}{use=structure,fg=brickred,bg=white}
\setbeamercolor{block body}{use=structure,fg=black,bg=white}
 

\usefonttheme{professionalfonts}
% default | professionalfonts | serif |	structurebold | structureitalicserif |structuresmallcapsserif
%\usepackage{eulervm}

%% User defined commands
\newcommand{\vect}[1]{\ensuremath{\mathbf{#1}}}
\newcommand{\trans}[1]{\ensuremath{#1}^{\footnotesize{\textsf{T}}}}
\newcommand{\calX}{\ensuremath{{\cal X}}}
\newcommand{\calY}{\ensuremath{{\cal Y}}}
\DeclareMathOperator{\sign}{sign}

%\newtheorem{theorem}{Theorem}
\title{Linear Regression}
\begin{document}

\maketitle

\begin{frame}[t]
  \frametitle{Linear Regression}  
\begin{itemize}
    \item Simple approach to supervised learning when the target variable is 
        \textbf{continuous}.
    \item Assumes a linear relationship between the input variables 
        $x_1, \ldots, x_n$ and the target variable $y$.
    \item  When the target variable is discrete, the problem is a
    \textbf{classification}
        problem. 
  \end{itemize}
\end{frame}

\begin{frame}[t]
\frametitle{Examples of Machine Learning}
\textcolor{orange2}{\textbf{Credit Approval}}

Applicant Information:

\begin{itemize}
    \item age
    \item gender
    \item salary
    \item current debt
    \item years in job $\ldots$ 
\end{itemize}

\pause

Should we approve credit? 

\bigskip

\textbf{Data}
\begin{itemize}
    \item Data on previous applications and their credit history.
\end{itemize}
\end{frame}

\begin{frame}[t]
\frametitle{Examples of Machine Learning}
\textcolor{orange2}{\textbf{Movie Ratings}}

Predict how a user would rate a movie.

\pause

\medskip

Each user is modelled as a vector of attributes:
\begin{itemize}
    \item likes comedy?  
    \item likes block-busters?
    \item likes sci-fi?
    \item likes a specific actor?
    \item $\ldots$
\end{itemize}

\pause

How would the user rate a given movie on a scale from $1$ to $10$?

\bigskip

\textbf{Data}
\begin{itemize}
    \item Data on how other users rated the given movie. 
\end{itemize}
\end{frame}


\begin{frame}[t]
\frametitle{Components of the Learning Problem}

\textbf{Formalization}

\begin{itemize}
    \item \textbf{Input:} $\vect{x}$ (applicant information)

    \pause

    \item \textbf{Output:} $y$ (good/bad customer)
    
    \pause

    \item \textbf{Target function:} $f \colon \calX \rightarrow \calY$ (ideal credit
    approval formula)
    
    \pause

    \item \textbf{Data:} $(\vect{x}^{(1)}, y^{(1)}), \ldots, (\vect{x}^{(m)},
    y^{(m)})$ (historical records)
    
    \pause

    \item \textbf{Hypothesis:} $h \colon \calX \rightarrow \calY$ (formula to be used)
\end{itemize} 
\end{frame}

\begin{frame}[t]
\frametitle{The Learning Problem}
\begin{center}
\includegraphics[scale=0.25]{the_learning_problem.png}
\end{center}
\end{frame}

\begin{frame}[t]
\frametitle{Solution Components}
Two solution components:
\begin{itemize}
    \item The Hypothesis Set $\cal{H}$
    \item The Learning Algorithm
\end{itemize}

Together, they are referred to as the \textbf{learning model}.

\pause

\bigskip

Why specify a hypothesis set?
\begin{itemize}
    \item This is what is generally done: you choose a linear model, 
    or an SVM or a neural network

    \item Important for developing a theory of learning 
\end{itemize}
\end{frame}

\begin{frame}[t]
\frametitle{Hypotheses Sets and Learning Algorithms}
\textbf{Examples}

\begin{center}
\begin{tabular}{ll}
\emph{Hypothesis Set} & \emph{Learning Algorithm} \\ \hline
Linear Regression & Gradient Descent \\
Neural Networks & Back Propagation \\
SVM & Quadratic Programming \\
Mixture of Gaussians Model & EM Algorithm \\
\end{tabular}
\end{center}
\end{frame}

\begin{frame}[t]
\frametitle{The Perceptron: A Simple Hypothesis Set}
For input $\vect{x} = (x_1, \ldots, x_n)$, the customer attributes,
\begin{align*}
\text{Approve credit if } & \sum_{i = 1}^{n} w_i x_i >    \text{ threshold} \\
\text{Deny credit if }    & \sum_{i = 1}^{n} w_i x_i \leq \text{ threshold}. \\
\end{align*}

This linear formula $g \in \cal{H}$ can be written as :
\[
    g(\vect{x}) = \sign \left (\sum_{i=1}^{n} w_i x_i - \text{ threshold} \right )
\]
\end{frame}

\begin{frame}[t]
\frametitle{The Perceptron: A Simple Hypothesis Set}
\[
    g(\vect{x}) = \sign \left (\sum_{i=1}^{n} w_i x_i - \text{ threshold} \right )
\]
\end{frame}

\begin{frame}[t]
\frametitle{The Perceptron: A Simple Hypothesis Set}
\[
    g(\vect{x}) = \sign \left (\sum_{i=1}^{n} w_i x_i + w_0\right )
\]

\pause

Introduce an artificial coordinate $x_0 = 1$:
\[
     g(\vect{x}) = \sign \sum_{i=0}^{n} w_i x_i = \sign(\trans{\vect{w}} \cdot \vect{x})
\]

\pause

\begin{center}
    \includegraphics[scale=0.25]{linearly_separable_data.png}
\end{center}
\end{frame}

\begin{frame}[t]
\frametitle{The Perceptron Learning Algorithm (PLA)}
The perceptron implements 
\[g(\vect{x}) = \sign(\trans{\vect{w}} \cdot \vect{x})\]

\parbox[t]{5cm}{
Given a training set: 
\[(\vect{x}^{(1)}, y^{(1)}), (\vect{x}^{(2)}, y^{(2)}), \ldots, (\vect{x}^{(m)},
y^{(m)})\]
pick a \textbf{misclassified} point:
\[
    \sign (\trans{\vect{w}} \vect{x}^{(k)}) \neq y^{(k)}
\]
and update the weight vector:
\[
    \vect{w}_{\text{new}} \leftarrow \vect{w}_{\text{old}} + y^{(k)} \vect{x}^{(k)}
\]
} 
\hfill
\parbox[t]{5cm}{
\begin{center}
    \includegraphics[scale=0.20]{PLA.png}
\end{center}
}
\end{frame}

\begin{frame}[t]
\frametitle{Iterations of PLA}
\parbox[t]{7cm}{
    \begin{itemize}
        \item One iteration of the PLA:
            \[ \vect{w}_{\text{new}} \leftarrow \vect{w}_{\text{old}} + y \vect{x}\]
        where $(\vect{x}, y)$ is a misclassified point.

        \item On iteration $i = 1, 2, 3, \ldots$, pick a misclassified point from 
            \[(\vect{x}^{(1)}, y^{(1)}), (\vect{x}^{(2)}, y^{(2)}), \ldots,
            (\vect{x}^{(m)}, y^{(m)})\]
        and run a PLA iteration on it. 
    \end{itemize}
}
\hfill
\parbox[t]{4cm}{
\begin{center}
    \includegraphics[scale=0.20]{PLA_iterations.png}
\end{center}
}

\begin{theorem}[Convergence]
If the data is linearly separable then the PLA will find a set of weights $\vect{w}$
that correctly classifies the training examples in a finite number of steps. 
\end{theorem}
\end{frame}

\begin{frame}[t]
\frametitle{Types of Learning}
\begin{itemize}
    \item Supervised Learning
    \item Unsupervised Learning
    \item Reinforced Learning 
\end{itemize}
\end{frame}

\begin{frame}[t]
\frametitle{Supervised Learning}
\begin{center}
\includegraphics[scale=0.22]{supervised_learning.png}
\end{center}
\end{frame}

\begin{frame}[t]
\frametitle{Unsupervised Learning}
\begin{center}
\includegraphics[scale=0.22]{unsupervised_learning.png}
\end{center}
\end{frame}

\begin{frame}[t]
\frametitle{Reinforced Learning}
\begin{center}
\includegraphics[scale=0.22]{reinforced_learning.png}
\end{center}
\end{frame}

\end{document}
