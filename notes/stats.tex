\chapter{Basic Statistics}
In this section, we review some basic concepts from Statistics. The material 
in this section is based on~\cite{DegSch}.

\section{Covariance and Correlation}

The \emph{correlation} between two sets of data is a measure of the strength of
the relationship between them. In particular, Pearson's correlation coefficient
is a measure of linear relationship between two sets of data. Let $X$ and $Y$
be two random variables. Then Pearson's correlation coefficient $\rho(X, Y)$ is
defined as:
\begin{equation}
    \rho(X, Y) = \frac{\cov(X, Y)}{\sd_X \sd_Y} 
\end{equation}  
Two important facts about the Pearson's correlation coefficient~\cite{CasBer}:
\begin{enumerate}
	\item $-1 \leq \rho(X, Y) \leq 1$
	\item $|\rho(X, Y)| = 1$ iff there exists $a \neq 0$ and $b$ such that 
	$Y = aX + b$.
\end{enumerate}

\section{Tail Inequalities and the Law of Large Numbers}
We first look at Markov and Chebyshev inequalities and then apply them to 
random samples.

The Markov inequality tells us how much probability mass can be at large 
values given the mean of the distribution.  
\begin{theorem}\label{thm:stats:markov_ineq}
Let $X$ be a random variable that takes on positive values only. Then for $t > 0$,
\begin{equation} \label{ineq:markov}
    \pr (X \geq t) \leq \frac{\E [X]}{t}.
\end{equation}
\end{theorem}
\begin{proof}
Let us assume that $X$ is a continuous r.v. with pdf $f_X$. By definition, 
\begin{align*}
    \E [X] & = \int_{0}^{\infty} x f_X(x) dx \\
           & = \int_{0}^{t} x f_X(x) dx + \int_{t}^{\infty} x f_X(x) dx \\
           & \geq t \int_{t}^{\infty} f_X(x) dx.
\end{align*} 
Since $t > 0$, dividing both sides of the last inequality yields the Markov
inequality.
\end{proof}
\noindent This inequality is useful when $t > \E [X]$; when $t \leq \E [X]$, it merely
bounds $\pr (X \geq t)$ by $1$. 

The Chebyschev inequality involves both the mean and the variance of the 
distribution. It bounds the probability of how far a random variable can be 
from its mean as a function of the variance. 
\begin{theorem}
Let $X$ be a random variable for which $\var (X)$ exists. Then 
\begin{equation} \label{ineq:chebyschev}
    \pr (\left | X - \E [X] \right | > t) \leq \frac{\var (X)}{t^2}.
\end{equation}
\end{theorem} 
\begin{proof}
Define $Y \coloneqq (X - \E [X])^2$ so that $\E [Y] = \var (X)$. Applying
the Markov inequality to the r.v.~$Y$, we obtain:
\[
    \pr (Y \geq t^2) \leq \frac{\var (X)}{t^2}.
\] 
But $\pr (Y \geq t^2) = \pr (| X - \E [X] | > t)$ and so this proves the 
Chebyschev inequality too.
\end{proof}

\section{Maximum Likelihood Estimation}
Let the random variables $X_1, \ldots, X_n$ form a random sample from a 
distribution with pdf~$f(x \mid \theta)$. Recall that this means that 
$X_i \iid f( \cdot \mid \theta)$ for all~$1 \leq i \leq n$. 
Let $f_n(\vec{x} \mid \theta)$ denote the value of the joint pdf of 
the random vector $(X_1, \ldots, X_n)'$ at the point 
$\vec{x} = (x_1, \ldots, x_n)'$. The \emph{likelihood function} is the joint pdf 
of the observations of a random sample viewed as a function of $\theta$ for 
a given set of values of the sample. The maximum likelihood estimate of 
$\theta$ is that value of $\theta$ for which $f_n (\vec{x} \mid \theta)$ is 
maximized. 
\begin{example}
Suppose that $X_1, \ldots, X_n$ form a random sample from a distribution with 
pdf $f(x \mid \theta)$ defined as follows:
\[
    f(x \mid \theta) = 
        \left \{
            \begin{array}{ll}
                \theta x^{\theta - 1} & \text{ for } 0 < x < 1 \\
                0                     & \text{ otherwise}.
            \end{array} 
        \right .
\]
Let us assume that $\theta > 0$. The joint distribution of the vector 
$(X_1, \ldots, X_n)'$ at the point $\vec{x} = (x_1, \ldots, x_n)'$ is 
\[
    f_n (\vec{x} \mid \theta ) = \prod_{i = 1}^n \theta x_i^{\theta - 1},
\]
where we assume that $0 < x_i < 1$ for all $i$. Taking logs, we obtain that
$\log f_n (\vec{x} \mid \theta) = n \log \theta + (\theta - 1) \sum_i \log x_i$. 
Take the derivative wrt~$\theta$ and set to $0$:
\[
    \frac{\partial \log f_n (\vec{x} \mid \theta)}{\partial \theta} = 
        \frac{n}{\theta} + \sum_i \log x_i \set 0
\]
to obtain that $\theta = n / \sum_i \log \frac{1}{x_i}$.
\end{example}   

\begin{example}
Suppose that $X_1, \ldots, X_n$ form a random sample from a distribution with 
pdf $f(x \mid \theta)$ defined as follows:
\[
    f(x \mid \theta) = \frac{1}{2} e^{- |x - \theta|} 
        \quad \text{ for } -\infty < x < \infty.
\]
Suppose that $\theta$ is unknown and that $-\infty < \theta < \infty$. 
In this case, the joint probability distribution is easily seen to be:
\[
    f_n(\vec{x} \mid \theta) = \frac{1}{2^n} e^{- \sum_i |x_i - \theta|}.
\]
Take logs to obtain: 
$\log f_n (\vec{x} \mid \theta) = \log \frac{1}{2^n} - \sum_i |x_i - \theta|$. 
Maximizing $f_n$ is equivalent to minimizing $\sum_i |x_i - \theta|$. This 
is equivalent to obtaining a point on the real line that minimizes the sum 
of the distances to the points $x_1, \ldots, x_n$. This happens when $\theta$
is the median of $x_1, \ldots, x_n$.
\end{example}

\begin{example}\label{ex:stats:uniform}
Suppose that $X_1, \ldots, X_n$ form a random sample from the uniform distribution
on the interval $[\theta_1, \theta_2]$, where both $\theta_1$ and $\theta_2$ 
are unknown $(-\infty < \theta_1 < \theta_2 < \infty)$. In this case, the 
log pdf of the joint distribution of $(X_1, \ldots, X_n)'$ is given by
\[
    \log f_n (\vec{x} \mid \theta_1, \theta_2) 
    = \log \prod_{i = 1}^n \frac{1}{\theta_2 - \theta_1}    
    = - n \log (\theta_2 - \theta_1).
\]
Maximizing the likelihood is equivalent to minimizing $\log (\theta_2 - \theta_1)$.
The minimum possible value of $\theta_2$ is $\max \{x_1, \ldots, x_n\}$
and the maximum possible value of $\theta_1$ is $\min \{x_1, \ldots, x_n\}$.
\end{example}

\begin{example}
Suppose that a certain large population contains $k$ different types of 
individuals $(k \geq 2)$, and let $\theta_i$ denote the proportion of people 
of type~$i$, for $1 \leq i \leq k$. Here, $0 \leq \theta_i \leq 1$ and 
$\sum_{i = 1}^k \theta_i = 1$. Suppose also that in a random sample of $n$ 
individuals from this population there are exactly $n_i$ individuals of type~$i$
so that $n = n_1 + \cdots + n_k$.

In this setting, for $1 \leq i \leq k$, define $X_i$ to be the number of 
individuals of type~$i$ in a random sample of size~$n$. Then the probability 
that $\bigwedge_{i = 1}^k X_i = n_i$ is given by $\theta_1^{n_1} \cdots \theta_k^{n_k}$.  
The log pdf of the joint distribution is given by:
\[
    \log f_k((n_1, \ldots, n_k) \mid \theta_1, \ldots, \theta_k, n) 
        = \sum_{i = 1}^k n_i \log \theta_i. 
\] 
Note that there are actually $k - 1$ variables here since we may write 
$\theta_k = 1 - \sum_{i = 1}^{k - 1} \theta_i$. Differentiating wrt $\theta_i$
for $1 \leq i \leq k - 1$, we obtain:
\[
    \frac{\partial \log f_k}{\partial \theta_i} 
        = \frac{n_i}{\theta_i} - \frac{n_k}{\theta_k}. 
\]
Setting this to $0$, we get that $\theta_i / \theta_k = n_i / n_k$. Sum this up
from $1 \leq i \leq k - 1$, to obtain:
\[
    \frac{1 - \theta_k}{\theta_k} = \frac{n - n_k}{n_k},
\] 
which yields $\theta_k = n_k / n$. Substitute this in $\theta_i / \theta_k = n_i / n_k$
to obtain $\theta_i = n_i / n$.
\end{example} 

\begin{example}[Nonexistence of an MLE]
An obvious disadvantage of the technique of maximum likelihood estimation 
is when the maximum does not exist. Consider again Example~\ref{ex:stats:uniform}
where we let $\theta_1 = 0$ and $\theta = \theta_2$
The pdf of the uniform distribution is defined as:
\[
    f(x \mid \theta) 
        = \left \{ 
            \begin{array}{ll}
                \frac{1}{\theta} & 0 \leq x \leq \theta \\
                0  & \text{ otherwise}. 
            \end{array}
          \right . 
\]
Let us modify the definition of the pdf so that we use strict inequalities 
$0 < x < \theta$ above. Given a sample $x_1, \ldots, x_n$, the log pdf is 
$- n \log \theta$ as before and the MLE technique would require us to 
minimize $\log \theta$. However, in this case, there is no  
$\theta > \max \{x_1, \ldots, x_n\}$ that mimimizes $\log \theta$ and the 
MLE does not exist.    
\end{example}

\begin{example}[Non-uniqueness of an MLE]
Consider a random sample $X_1, \ldots, X_n$ from the uniform distribution 
over the interval $[\theta, \theta + 1]$. In this case, the joint pdf 
is given by:
\[
    f_n(\vec{x} \mid \theta) 
        = \left \{ 
            \begin{array}{ll}
                1  & \text{ for } \theta \leq x_i \leq \theta + 1 \quad (1 \leq i \leq n)\\
                0  & \text{ otherwise}. 
            \end{array}
          \right . 
\]
In this case, the condition $\theta \leq x_i \leq \theta + 1$ for $1 \leq i \leq n$
may be written using the two conditions:
\[
    \theta \leq \min \{x_1, \ldots, x_n\} \text{ and } 
    \max \{x_1, \ldots, x_n \} - 1 \leq \theta.
\]
Any value of $\theta$ in the interval $[\max \{x_1, \ldots, x_n \} - 1, 
\min \{x_1, \ldots, x_n\}]$ is valid, but there is no unique value of $\theta$.
\end{example}

\section{Bayesian Statistics and MCMC}
This section is based on Chapters~12--15 from~\cite{Lam}. Bayes' rule gives us a
 recipe for calculating the posterior probability density. 
\begin{equation}
	\pr (\Theta \mid \text{data}) = 
	\frac{\pr (\text{data} \mid \Theta) \cdot \pr (\Theta)}{\pr (\text{data})}.
\end{equation}
Consider a case in which we have a sample of $N$ data points $x_1, \ldots, x_N$. 
We assume that the likelihood is a Poisson distribution with mean $\lambda$ and 
that the prior for $\lambda$ is a log-normal$(1, 1)$ distribution. To calculate 
the probability of the data $\pr (\text{data})$, we must evaluate the integral:
\begin{equation}
	\pr (\text{data}) = \int_{0}^{\infty} 
		\prod_{i = 1}^N \frac{\lambda^{x_i} e^{- \lambda}}{x_i!} \cdot 
		\frac{1}{\sqrt{2 \pi} \lambda} e^{- \frac{1}{2} (\log \lambda - 1)^2} 
		d\lambda.
\end{equation}
While this integral is not too difficult, it explains the problem of calculating 
posteriors analytically. As the number of parameters (the length of $\Theta$) 
increases, calculating the probability of the data requires evaluating integrals 
in higher dimensional spaces. This is why we use alternative methods to derive 
approximate versions of the posterior.

